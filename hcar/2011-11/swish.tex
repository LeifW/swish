% Swish-DS.tex
\begin{hcarentry}[new]{Swish}
\report{Douglas Burke}%11/11
\status{experimental}
\participants{Graham Klyne, Vasili I Galchin}
\makeheader

Swish is a framework for performing deductions in RDF data using a
variety of techniques. Swish is conceived as a toolkit for
experimenting with RDF inference, and for implementing stand-alone RDF
file processors (usable in similar style to CWM, but with a view to
being extensible in declarative style through added Haskell function
and data value declarations). It explores Haskell as ``a scripting
language for the Semantic Web'', is a work-in-progress, and currently
incorporates:

\begin{compactitem}
\item
Support for Turtle, Notation3, and NTriples formats.

\item
RDF graph isomorphism testing and merging.

\item
Display of differences between RDF graphs.

\item
Inference operations in forward chaining, backward chaining and proof-checking modes.

\item
Simple Horn-style rule implementations, extendable through variable binding modifiers and filters.

\item
Class restriction rule implementation, primarily for datatype inferences.

\item
RDF formal semantics entailment rule implementation.

\item
Complete, ready-to-run, command-line and script-driven programs.
\end{compactitem}

\subsubsection*{Current Work}

A number of incremental changes have been made to the code base,
including support for version 7.2 of GHC and some minor
optimisations. A parser and formatter for the Turtle format were
added, the API changed to use the Text datatype where appropriate, and
the vocabulary module was extended to include terms from the Dublin
Core, FOAF, Geo and SIOC vocabularies.

\FuturePlans

Continue the clean up and replacement of code with packages from
Hacakge. Look for commonalities with the other existing RDF Haskell package,
{\tt rdf4h}.
Community input --- whether it be patches, new code or just feature
requests --- are more than welcome.
 
\FurtherReading
\begin{compactitem}
\item \url{https://bitbucket.org/doug_burke/swish/}
\item \url{http://www.ninebynine.org/RDFNotes/Swish/Intro.html}
\item \url{http://protempore.net/rdf4h/}
\end{compactitem}
\end{hcarentry}
