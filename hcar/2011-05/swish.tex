\documentclass{scrreprt}
\usepackage{paralist}
\usepackage{graphicx}
\usepackage[final]{hcar}

%include polycode.fmt

\begin{document}

\begin{hcarentry}{Swish}
\report{Douglas Burke}
\status{Experimental}
\participants{Graham Klyne, Vasili I Galchin}
\makeheader

Swish is a framework, written in the purely functional programming
language Haskell, for performing deductions in RDF data using a
variety of techniques. Swish is conceived as a toolkit for
experimenting with RDF inference, and for implementing stand-alone RDF
file processors (usable in similar style to CWM, but with a view to
being extensible in declarative style through added Haskell function
and data value declarations). It explores Haskell as ``a scripting
language for the Semantic Web'', is a work-in-progress, and currently
incorporates:

\begin{itemize}

\item
Support for both Notation3 and NTriples formats.

\item
RDF graph isomorphism testing and merging.

\item
Display of differences between RDF graphs.

\item
Inference operations in forward chaining, backward chaining and proof-checking modes.

\item
Simple Horn-style rule implementations, extendable through variable binding modifiers and filters.

\item
Class restriction rule implementation, primarily for datatype inferences.

\item
RDF formal semantics entailment rule implementation.

\item
Complete, ready-to-run, command-line and script-driven programs.

\end{itemize}

\subsubsection*{Current Work}

The version on Hackage has recently been updated from 0.2.1 to
the 0.3 series; the main changes were to make the package build
with recent Haskell Platform releases, updates to match the latest
N3 specification, and addition of the NTriples format. Minor bug fixes
and improvements have been made to this series.

\subsubsection*{Future Plans}

The major planned changes are a move to using the
Data.Text module,
addition of an RDF/XML parser, profiling 
and further clean up of the code.
Community input -- whether it be patches, new code or just feature
requests -- are more than welcome.

\FurtherReading

\begin{itemize}
\item \url{https://bitbucket.org/doug_burke/swish/}
\item \url{http://www.ninebynine.org/RDFNotes/Swish/Intro.html}
\end{itemize}

\end{hcarentry}

\end{document}
